\documentclass[a4paper]{article}
%\usepackage{ctex}
\usepackage{indentfirst}
\usepackage{longtable}
\usepackage{multirow}
%\setlength{\parskip}{0.0em}
\renewcommand{\baselinestretch}{1.5} \normalsize

\usepackage{CJK}
%\usepackage[margin=1in]{geometry}
\usepackage[fleqn]{amsmath}
\usepackage{parskip}
\usepackage{listings}
\setlength{\parindent}{0em}

\usepackage{graphicx}
\usepackage{float}
\usepackage{multicol}
\usepackage{amssymb}
\usepackage{epstopdf}
\begin{document}
\begin{CJK*}{GBK}{song}
\title{\textbf{\fontsize{20pt}{\baselineskip}\selectfont 离散数学作业 Problem set 4}}
\date{}\maketitle
\section*{Problem 1}
计算下列集合的基数.
\begin{description}
\item (1) $A=\{x,y,z\} $
\item (2) $B=\{x|x=n^2\wedge n\in N\}$
\item (3) $C=\{x|x=n^{109}\wedge n\in N\}$
\item (4) $B\cap C$
\item (5) $B\cup C$
\item (6) 平面上所有的圆心在$x$轴上的单位圆的集合.
\end{description}

\section*{Problem 2}
确定下列各集合是否是有限的、可数无限的或不可数的。对那些可数无限集合,给出在自然数集合和该集合之间的一一对应。

\begin{description}
\item a) 大于10的整数
\item b) 奇负整数
\item c) 绝对值小于1\,000\,000的整数
\item d) 0和2之间的实数
\item e) 集合$A\times Z^+$这里$A=\{2,3\}$
\item f) 10 的整数倍
\end{description}

\section*{Problem 3}
假设$A$是可数集合。证明如果存在一个从$A$到$B$的满射函数$f$,则$B$也是可数的。

\section*{Problem 4}
证明:任取8个自然数,必有两个数的差是7的倍数。

\section*{Problem 5}
设$A=\{a,b,c\}$,$B=\{0,1\}^A$,由定义证明$\mathcal P(A)\approx \{0,1\}^A$.

\section*{Problem 6}
设$a,b,c,d$均为正整数,下列命题是否为真?若为真,给出证明;否则,给出反例.
\begin{multicols}{2}
\begin{description}
\item a) 若$a \mid c$, $b \mid c$, 则$ab \mid c$.
\item c) 若$ab \mid c$, 则$a \mid c$.
\item b) 若$a \mid c$, $b \mid d$, 则$ab \mid cd$.
\item d) 若$a \mid bc$, 则$a \mid b$或$a \mid c$.
\end{description}
\end{multicols}

\section*{Problem 7}
证明:若$n$和$k$为正整数,则有$\lceil n/k\rceil=\lfloor (n-1)/k\rfloor+1$。

\section*{Problem 8}
计算:
\begin{multicols}{3}
\begin{description}
\item a) 23300 mod 11
\item b) $2^{3300}$ mod 31
\item c) $3^{516}$ mod 7
\end{description}
\end{multicols}

\section*{Problem 9}
证明:存在无穷多个$n$使得$\phi(n)>\phi(n+1)$.

\section*{Problem 10}
证明:若$m$和$n$互素,则$m^{\phi(n)}+n^{\phi(m)}\equiv$1(mod $mn$).

\end{document} 
