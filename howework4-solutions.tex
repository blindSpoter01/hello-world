\documentclass[a4paper]{article}
%\usepackage{ctex}
\usepackage{indentfirst}
\usepackage{longtable}
\usepackage{multirow}
%\setlength{\parskip}{0.0em}
\renewcommand{\baselinestretch}{1.5} \normalsize

\usepackage{CJK}
%\usepackage[margin=1in]{geometry}
\usepackage[fleqn]{amsmath}
\usepackage{parskip}
\usepackage{listings}
\setlength{\parindent}{0em}

\usepackage{graphicx}
\usepackage{float}
\usepackage{multicol}
\usepackage{amssymb}
\begin{document}
\begin{CJK*}{GBK}{song}
\title{\textbf{\fontsize{20pt}{\baselineskip}\selectfont 离散数学第四次作业参考答案}}
\date{}\maketitle
\section*{Problem 1}
\begin{multicols}{5}
\begin{description}
\item (1) $3 $
\item (2) $\aleph_0$
\item (3) $\aleph_0$
\item (4) $\aleph_0$
\item (5) $\aleph_0$
\item (6) $\aleph$

\end{description}
\end{multicols}

\section*{Problem 2}
确定下列各集合是否是有限的、可数无限的或不可数的。对那些可数无限集合,给出在自然数集合和该集合之间的一一对应。
\begin{description}
\item a) 大于10的整数
\item b) 奇负整数
\item c) 绝对值小于1\,000\,000的整数
\item d) 0和2之间的实数
\item e) 集合$A\times Z^+$这里$A=\{2,3\}$
\item f) 10 的整数倍
\end{description}
\subsection*{Solution.}
\begin{description}
\item (1) 可数无限集,$n\leftrightarrow n+11$。
\item (2) 可数无限集,$n\leftrightarrow -2n-1$
\item (3) 有限集。
\item (4) 不可数集。
\item (5) 可数无限集。n为奇数,则$n\leftrightarrow (3, \dfrac{n+1}{2})$;n为偶数,则$n\leftrightarrow (2, \dfrac{n}{2}+1)$
\item (5) 可数无限集,$n\leftrightarrow 10\times (-1)^{n}\times \lfloor \dfrac{n+1}{2}\rfloor$。
\end{description}

\section*{Problem 3}
假设$A$是可数集合。证明如果存在一个从$A$到$B$的满射函数$f$,则$B$也是可数的。
\subsection*{Solution.}
\begin{description}
\item (1) $A=\emptyset$且$B=\emptyset$,易证。
\item (2) $f:A\rightarrow B$是满射函数,易得$card\,A\geq card\,B$,又因为A是可数的,则B为可数的。
\end{description}

\section*{Problem 4}
证明:任取8个自然数,必有两个数的差是7的倍数。
\subsection*{Solution.}
在与整除有关的问题中有这样的性质,如果两个整数a、b,它们除以自然数m的余数相同,
那么它们的差a-b是m的倍数。根据这个性质,本题只需证明这8个自然数中有2个自然数,
它们除以7的余数相同。我们可以把所有自然数按被7除所得的7种不同的余数0、1、2、3、4、5、6分成七类.也就是7个抽屉。
任取8个自然数,
根据抽屉原理,必有两个数在同一个抽屉中,也就是它们除以7的余数相同,因此这两个数的差一定是7的倍数。

\section*{Problem 5}
$\mathcal P (A)=\{\O,\{a\},\{b\},\{c\},\{a,b\},\{a,c\},\{b,c\},\{a,b,c\}\}$,$\{0,1\}^A=\{f_0,f_1,f_2,f_3,f_4,f_5,f_6,f_7 \}$,构造双射函数$f$.\\
$f(\O)=f_0$,$f(\{a\})=f_1$,$f(\{b\})=f_2$,$f(\{c\})=f_3$,\\
$f(\{a,b\})=f_4$,$f(\{a,c\})=f_5$,$f(\{b,c\})=f_6$,$f(\{a,b,c\})=f_7$,\\
根据等势的定义有$\mathcal P(A)\approx \{0,1\}^A$.请注意题意要求等势定义证明.

\section*{Problem 6}
a) 假。\\
b) 真。证明:由题设,存在整数$k_1$,$k_2$使得$c=k_1 a,d=k_2 b$,从而有$cd=k_1 k_2 ab$,得证$ab|cd$.\\
c) 真。证明:存在整数$k$使得$c=k(ab)=(kb)a$,得证$a|c$。\\
d) 假。

\section*{Problem 7}
证明:若$n$和$k$为正整数,则有$\lceil n/k\rceil=\lfloor (n-1)/k\rfloor+1$。
\subsection*{Solution.}
令$n=ka+b, 0\leq b<a$。

若$b=0$,则$\lceil n/k\rceil=a$,$\lfloor (n-1)/k\rfloor=a-1$. 

若$b\geq 1$,则$\lceil n/k\rceil=a+1$,$\lfloor (n-1)/k\rfloor=a$.

\section*{Problem 8}
计算:
\begin{multicols}{3}
\begin{description}
\item a) 23300 mod 11
\item b) $2^{3300}$ mod 31
\item c) $3^{516}$ mod 7
\end{description}
\end{multicols}
\subsection*{Solution.}
a)$23300=233*100=(21*11+2)*(9*11+1)\equiv2*1 \equiv2 \pmod {11}=2$

b)$2^{3300}=2^ {5*660}\equiv32^{660}\equiv1^{660}\equiv1 \pmod {31}=1$

c)$3^6\equiv1 \pmod 7, 3^{516}\equiv3^{6*86}\equiv1 \pmod 7=1$

\section*{Problem 9}

设 $n$ 是大于3的素数,$\phi(n)=n-1$. 又 $n+1$是偶数,有$\frac{n-1}{2}$个小于$n+1$的偶数,它们均与$n+1$不互素,故$\phi(n+1)=n-1\leq n-\frac{n-1}{2}=\frac{n+1}{2}=\phi(n)$. 而大于3的素数有无穷多个,证毕.

\section*{Problem 10}

由欧拉定理,$m^\phi(n)\equiv 1(mod~~n)$,即$n|  m^{\phi(n)} -1$. 同理$m|  n^\phi(m) -1$.
从而,$mn|(m^\phi(n) -1)(n^\phi(m) -1)$,即$mn| m^\phi(n) n^\phi(m) -(m^\phi(n) +n\phi(m) -1)$. 而$mn |  m^\phi(n)  n^\phi(m)$ ,故有$mn|(m^\phi(n) +n^\phi(m) -1$,
得证$m^\phi(n) +n^\phi(m) \equiv 1(mod~~n)$.

\end{document} 